%Dokumentinformationen
\newcommand{\titleinfo}{Informatik - Zusammenfassung}
\newcommand{\authorinfo}{C.Renda, R. von Reding}
\newcommand{\versioninfo}{$Revision: $ \today}

%Weitere Autoren %%%%%%%%%%%%%%%%%%%%%%%%%%%%%%%%%%%%%%%%%%%%%%%%%%%%%%%%%%%%%%%%%%%%%%
%C.Renda

% standard header
%Schriftgr�sse, Layout, Papierformat, Art des Dokumentes
\documentclass[8pt,twoside,a4paper,fleqn]{extarticle}
\usepackage{extsizes}
%Einstellungen der Seitenraender
\usepackage[left=0.5cm,right=0.5cm,top=0.5cm,bottom=0.5cm,includeheadfoot,landscape=true]{geometry}
% Sprache, Zeichensatz, packages
%\usepackage[latin1]{inputenc}
\usepackage[english,ngerman]{babel}
\usepackage[utf8]{inputenc}
\usepackage{cancel}

%\usepackage[ngerman]{babel,varioref}
\usepackage{amssymb,amsmath,fancybox,graphicx,color,lastpage,wrapfig,fancyhdr,hyperref,verbatim}
\usepackage[T1]{fontenc}

\newcommand{\licence}{CC BY-NC-SA}

%pdf info
\hypersetup{pdfauthor={\authorinfo},pdftitle={\titleinfo},colorlinks=false}
%linkbordercolor=white
\author{\authorinfo}
\title{\titleinfo}

%Kopf- und Fusszeile
\pagestyle{fancy}
\fancyhf{}
%Linien oben und unten
\renewcommand{\headrulewidth}{0.5pt} 
%\renewcommand{\footrulewidth}{0.5pt}

\fancyhead[L]{\titleinfo{ }\tiny{(\versioninfo)}}
%Kopfzeile rechts bzw. aussen
\fancyhead[R]{Seite \thepage { }von \pageref{LastPage}}

%Fusszeile
%\fancyfoot[L]{\footnotesize{\authorinfo}}
%\fancyfoot[R]{\footnotesize{\today}}
%\fancyfoot[C]{\footnotesize{\licence \quad $\rightarrow$ \href{https://github.com/HSR-Stud}{Github: HSR-Stud}}}

%Programmausschnitte

\usepackage{listings}
%\usepackage{lstlinebgrd}

\definecolor{bgGray}{rgb}{0.95,0.95,0.95}
\definecolor{stringColor}{rgb}{0.16,0.00,1.00}
\definecolor{annotationColor}{rgb}{0.39,0.39,0.39}
\definecolor{keywordColor}{rgb}{0.50,0.00,0.33}
\definecolor{commentColor}{rgb}{0.25,0.50,0.37}

\lstdefinestyle{cpp}{ 
	%linebackgroundcolor={\ifodd\value{lstnumber}\color{bgGray}\else\color{white}\fi},   % choose the background color; you must add \usepackage{color} or \usepackage{xcolor}; should come as last argument
	backgroundcolor=\color{bgGray},
	basicstyle=\normalsize\ttfamily,        % the size of the fonts that are used for the code
	breakatwhitespace=false,         % sets if automatic breaks should only happen at whitespace
	breaklines=true,                 % sets automatic line breaking
	captionpos=none,                    % sets the caption-position to bottom
	commentstyle=\color{commentColor},    % comment style
	deletekeywords={...},            % if you want to delete keywords from the given language
	escapeinside={\%*}{*)},          % if you want to add LaTeX within your code
	extendedchars=true,              % lets you use non-ASCII characters; for 8-bits encodings only, does not work with UTF-8
	frame=tb,	                  	 % adds a frame around the code
	keepspaces=true,                 % keeps spaces in text, useful for keeping indentation of code (possibly needs columns=flexible)
	keywordstyle=\color{keywordColor}\bfseries,   % keyword style
	language=C++,                    % the language of the code
	morekeywords={*,...},            % if you want to add more keywords to the set
	numbers=none,                    % where to put the line-numbers; possible values are (none, left, right)
	numbersep=3pt,                   % how far the line-numbers are from the code
	numberstyle=\footnotesize\color{codeGray}, % the style that is used for the line-numbers
	rulecolor=\color{black},         % if not set, the frame-color may be changed on line-breaks within not-black text (e.g. comments (green here))
	showspaces=false,                % show spaces everywhere adding particular underscores; it overrides 'showstringspaces'
	showstringspaces=false,          % underline spaces within strings only
	showtabs=false,                  % show tabs within strings adding particular underscores
	stringstyle=\color{stringColor},     % string literal style
	tabsize=2,	                   % sets default tabsize to 2 spaces
	title=\lstname                   % show the filename of files included with \lstinputlisting; also try caption instead of title
}


 % ./header.tex nicht editieren (Projekt LaTeX-Header benutzen)

%%%%%%%%%%%%%%%%%%%%%%%%%%%%%%%%%%%%%%%%%%%%%%%%%%%%%%%%%%%%%%%%%%%%%%%%%%%%%%%%%%%%%%%%%%%%%%%%
% Neue Befehle und Definitionen                
%%%%%%%%%%%%%%%%%%%%%%%%%%%%%%%%%%%%%%%%%%%%%%%%%%%%%%%%%%%%%%%%%%%%%%%%%%%%%%%%%%%%%%%%%%%%%%%
% This is needed for one more subsection, ex. 1.1.1.1, is called by \paragraph{}
\usepackage{titlesec}
\setcounter{secnumdepth}{4}
\setcounter{tocdepth}{4}
\titleformat{\paragraph}
{\normalfont\normalsize\bfseries}{\theparagraph}{1em}{}
% Settings which are used to set the distance above and under the sections
%\titlespacing*{\paragraph}{0pt}{2.25ex plus 1ex minus .2ex}{1.0ex plus .2ex}
\titlespacing{\section}{0em}{0.5em}{0.5em}
\titlespacing{\subsection}{0em}{0.5em}{0.5em}
\titlespacing{\subsubsection}{0em}{0.5em}{0.5em}

% Linksbündig
\setlength\parindent{0ex}

% This is needed for a smaller itemlist, is called by \compactenum {}
\usepackage{paralist}

% This is needed for merging some columns in a table
\usepackage{multicol} 
\usepackage{multirow}
\usepackage{array}
\usepackage{tabularx}


% This is needed for code listing
\usepackage[]{listings}
\lstset{literate=%
	{Ö}{{\"O}}1
	{Ä}{{\"A}}1
	{Ü}{{\"U}}1
	{ß}{{\ss}}1
	{ü}{{\"u}}1
	{ä}{{\"a}}1
	{ö}{{\"o}}1
	{~}{{\textasciitilde}}1
}

% This is needed to include Graphics
\usepackage{graphicx}

% This is needed for UML Diagrams
\usepackage{tikz}
\usepackage{pgf-umlcd}

% Courier font
\usepackage{courier}

% Todo Notes
\usepackage{todonotes}

\definecolor{red}{rgb}{1,0,0}
\definecolor{blue}{rgb}{0,0,1}
\definecolor{black}{rgb}{0,0,0}
\newcommand{\verweisc}[1]{$_{\textcolor{red}{\mbox{\small{C Kap. #1}}}}$}
\newcommand{\verweiscpp}[1]{$_{\textcolor{blue}{\mbox{\small{C++ Kap. #1}}}}$}
\newcommand{\verweisboth}[2]{$_{\textcolor{red}{\mbox{\small{C Kap. #1}}}}$$_{\textcolor{black}{\mbox{\small{, }}}}$$_{\textcolor{blue}{\mbox{\small{C++ Kap. #2}}}}$}
\newcommand{\verweishoch}[1]{${\textcolor{red}{\mbox{\small{Kapitel #1}}}}$}
\newcommand{\lc}[1]{\textit{\texttt{#1}}}

%Document Anfang
\begin{document}	


	\raggedbottom
	\lstset{style=cpp}
\begin{multicols*}{4}
	\section{Aufbau eines Programmes}

\begin{lstlisting}
#include <iostream> // Standart In-/ Output stream
#include <vector>		// Vector library
#include <cmath>		// Für math. funktionen	
#include <time.h>		// Zeitmessung
#include "headerfile.h" // Einbiden Headerfile
#define N 10	// defines jeglicher art

//structs, functions, enums

int main(void)
{
//programm code
return 0;
}
\end{lstlisting}


	\section{Variablen}

\begin{center}
	\begin{tabular}{ |l|l|l| } 
		\hline
		 \texttt{char} & 1 byte  8 bits. \\ 
		 \texttt{char16\_t} & At least 16 bits \\ 
		 \texttt{char32\_t} & At least 32 bits. \\ 
		\hline
		
		 \texttt{signed char} 			& Min 8 bits. \\ 
		 \texttt{signed short int} 	& Min 16 bits. \\ 
		 \texttt{signed int} 			& Min 16 bits. \\ 
		 \texttt{signed long int} 		& Min 32 bits. \\
		 \texttt{signed long long int} & Min 64 bits. \\
		\hline
		
 		
		 \texttt{unsigned char} 			& Min 8 bits. \\ 
		 \texttt{" short int} 		& Min 16 bits. \\ 
		 \texttt{"int} 			& Min 16 bits. \\ 
		 \texttt{" long int} 		& Min 32 bits. \\
		 \texttt{" long long int} 	& Min 64 bits. \\
		\hline
	\end{tabular}
\end{center}
\subsection{Variablennamen}
Keine Leerzeichen, Satzzeichen oder \_ Symbole Keine Zahl oder am Anfang case sensitivity – Gross - Kleinschreibung beachten

\subsection{Einfache Variablen deklarieren}
\begin{lstlisting}
int a,a2;   int b (1);
int b = 10;  int b {1};
float c = a*b - 0.5;
\end{lstlisting}
\subsection{Casts}
Änderung einer Variable in einen anderen Type
\begin{lstlisting}	
double a = 1.5; int b;
b = int (a);
b = (int) a; // b=1
7/2 = 3 , 7/(double)2 = 7/2.0 = 3.5
double(7/2) = 3.0 , int(19/10.0) = 1
\end{lstlisting}


\subsection{Enum}
Enum ist ein Aufzählungstyp. Die Konstanten aus der Enum
kann man im Programm verwenden.
\begin{lstlisting}	
enum farbe {ROT, BLAU, GELB};
farbe f = ROT;
if(f != BLAU) { }; 
\end{lstlisting}

\subsection{Hexadezimaler Code \& Adressen}
0,1,...,9,A,B,C,D,E,F (hex) anstelle von 0,1,...,14,15,16 (dec)
Adressen werden hexadezimal angegeben. $a,a+1,a+2,a+3,...$

\begin{center}
	\begin{tabular}{ ll } 
		\hline
int,float(4byte) & double (8byte)\\
\hline
0x22ff70 & 0x22ff70\\
0x22ff74 & 0x22ff78\\
0x22ff78 & 0x22ff80\\
		\hline
	\end{tabular}
\end{center}

\subsection{Fliesskommazahlen}
\begin{center}
	\begin{tabular}{ ll } 
		float&1b$\rightarrow$sign, 8b$\rightarrow$exp, 23b$\rightarrow$mantisse\\
		&Wert = $(-1)^{\texttt{S}} \cdot 2^{(\texttt{E-127})} \cdot (1.\texttt{F})$\\
		&Bsp: $0.125 = 2^{3} \Rightarrow \texttt{S} \rightarrow 0, \texttt{E} \rightarrow 124, \texttt{F} \rightarrow 0$\\
		\hline
		 & 0 | 01111100 | 00000000...0 = 0.125\\
		 & 0 | 01111111 | 00000000...0 = 1\\
		 & 1 | 01111111 | 11000000...0 = -1.75\\
		 & 0 | 00000000 | 00000000...0 = 0\\
		 & 0 | 11111111 | 00000000...0 = +infty\\
		 & 0 | 00000000 | 00000000...0 = NaN\\
		\hline
		double& 1b$\rightarrow$sign, 11b$\rightarrow$exp, 52b$\rightarrow$mantisse\\
		& Wert = $(-1)^{\texttt{S}} \cdot 2^{(\texttt{E-1023})} \cdot (1.\texttt{F})$\\
	\end{tabular}
\end{center}
















	\section{Operatoren}

\begin{center}
	\begin{tabular}{ ll } 
		$+$ $-$ $\ast$ $/$ $\hat{}$ & mathematische Operatoren\\
		 \% & Modulo\\
		x $+=$ i; & x $=$ x $+$ i; ebenso $*=$, $/=$, $-=$\\
		1.1E$^{-5}$ & = $1.1\cdot10^{-5}$\\
		i$++$, i$--$ & erhöt / verkleinert i um 1
	\end{tabular}
\end{center}
b$=$5; c$=$b++;$\rightarrow$c$=$5,b$=$6 verwende $++$b für c$=$6,b$=$6

Für weitere mathematische Funktionen
\begin{lstlisting}
#include <cmath>
fabs(), sqrt(), exp(), log(), cos(),
acos()
\end{lstlisting}
















	\section{Logische Konstrukte}

\begin{center}
	\begin{tabular}{ ll } 
		$<$, $<=$, $>$, $>=$ & grösser, grössergleich, kleiner\\
		|| & oder\\
		\&\& & und\\
		$==$ & gleichheit\\
		$!=$ ungleich\\
		! & nicht\\
		
	\end{tabular}
\end{center}



























	\section{iostream}
\begin{lstlisting}
using namespace std;
cout << "a =" << endl; //Ausgabe
cin >> a; //Eingabe
"\n" //Zeilenende "\t" //tabulator
"\"" //Anführungszeichen
\end{lstlisting}




























	\section{Umrechnung Binär-Dezimal}


\begin{center}
	\begin{tabular}{ lll } 
	13	&	1	& Dezimalzahl durch 2 teilen und rest\\
	6	& 	0	& notieren. Bits von unten nach oben\\
	3	&	1	& lesen.\\
	1	&	1	&6 bsp: 13 = 1101\\
	0 & & \\
		
	\end{tabular}
\end{center}
1001 = $1\cdot2^3 + 0 \cdot 2^2 + 0 \cdot 2^1 + 1 \cdot 2^0 = 8+0+0+1=9$
























	\section{Kontrollstrukturen}
\textbf{if()}
\begin{lstlisting}
if(a==10){
b=15;
}
else if(a==11) b=14;
else b=10;
// oder kurz
a==10?b=15:a==11?b=14:b=10;
\end{lstlisting}

\textbf{for()}
\begin{lstlisting}
for(int i=0; i<10; i++)
{
a=a+i;
} // abbruch mit break;
\end{lstlisting}

\textbf{while()}
\begin{multicols}{2}
	\begin{lstlisting}
while(b<20){
b++;
}	
	\end{lstlisting}
oder
	\begin{lstlisting}
do {
b--; //min 1 x
}
while(b!=15);
	\end{lstlisting}
\end{multicols}
abbrechen mit break (immer nur die innere Schleife)
Überspringen des Rests des Rumpfes zur nächsten
Auswärtung mit continue;


\textbf{switch case()}
\begin{lstlisting}
switch(a) {
case 15:cout<<"a=15";break;
case 14:cout<<"a=14";break; //a==14
default:cout<<"a!=15,a!=14"; //else
}
\end{lstlisting}

\textbf{Äquivalente Strukturen}
Man kann verschiedene Strukturen verwenden um ein und dasselbe auszudrücken: 
\begin{lstlisting}
int i=0;
do {
i=i+1;
if (i==10) break;
}while(true); //aka immer
\end{lstlisting}
...ist äquivalent zu... 
\begin{lstlisting}
for(int i=0;i!=10;i++){} 
\end{lstlisting}


\textbf{Endlosschleifen}
Man kann verschiedene Strukturen verwenden um ein und dasselbe auszudrücken: 
\begin{lstlisting}
int i=10;
do {
i=i+1;
if (i==10) break;
}while(true);
for(int i=3;i!=20;i=(i+3)%300)
int i=99;
while(i>10){
i--;
if(i==15) i*=6;} 
\end{lstlisting}





	\section{Arrays}
\begin{lstlisting}
int a[4]; // int array mit 4 Zellen
a[0] = 1; // Definition des 1. Elements
int a[4] = {1,2,3,4}; 
int b[3][2] = {{1,2},{3,4},{5,6}};
int c[x][3][x-1] = {{{ 
\end{lstlisting}

\begin{itemize}
	\item Bei int a[N]; muss N als const int N = 10; definiert werden. Eine const int kann wäred dem Programmablauf nicht geändert werden.  
	\item Ein Array beginnt immer mit a[0] und  endet mit a[N-1]
	\item Übergibt man ein Array einer Funktion, ist das wie "Call by reference". Das Original-Array wird verändert.
	
\end{itemize}

\subsection{Arrays und Pointer}
\textbf{Arraynamen sind Pointer!}

Bei der Definition eines Arrays wird Speicherplatz für eine bestimmte Anzahl Objekte reserviert. Die Arrayvariable zeigt auf das erste Objekt dieses Speicherplatzes. Darum sind folgende Ausdrücke identisch:

\begin{lstlisting}[mathescape]
int c[10]; 	// Array definieren
int* pc; 	// Pointer definieren 
pc = c;		// Pointer zeigt auf Array
pc[3] = 10;	$\leftrightarrow$ c[3]=10; $\leftrightarrow$ *(pc+3)=10;
\end{lstlisting}

Folgendes generiert auch ein Array mit Platz für 3
Integer:

\begin{lstlisting}[mathescape]

int* a = new int[3];
a[0] =3;		// ohne Stern *
delete [] a; 	// Speicher wird wieder freigegeben
\end{lstlisting}

	\section{Structures}

Structs werden vor der main() Funktion definiert.
\begin{lstlisting}[mathescape]
struct point {
int x, y;
double gamma;
}p,q; 	// p,q schon definiert

point k; 	// Neuer "point" definieren
k.x = 2;	// Variable in struct definieren
k = {1,2,0.75} // schnell initialisieren
q = {1} $\leftrightarrow$ q ={1,0,0} // rest wird mit 0 aufgefüllt
p = q; // ist gleich wie
p.x = q.x; 
p.y = q.y;
p.gamma = q.gamma;
\end{lstlisting}

\textbf{Falsch Rekursion:} 
\begin{lstlisting}[mathescape]
struct point {int x; point y;}; // Keine Rekursion
\end{lstlisting}
\textbf{Richtig Selbstreferenzierung:} 
\begin{lstlisting}[mathescape]
struct node
{
int data;
struct node *next; // <-self reference
};
\end{lstlisting}
\textbf{Strichpunkt am Ende nicht vergessen} 
\texttt{struct point \{int i; double y;\};}

\subsection{Funktionen in Structs}
Die Konstruktor-Funktion wird bei der Generierung eines neuen Structs aufgerufen.
\begin{lstlisting}[mathescape]
struct Bar
{
	Bar() {//Konstruktor}
};
\end{lstlisting}
Funktionen können auch ausgelagert werden:
\begin{lstlisting}[mathescape]
struct Bar
{
void bier();
}bqm;
void Bar::bier() {};
\end{lstlisting}
Aufrufen der Funktion: \texttt{bqm.bier();}



























	\section{Dynamische Datenstrukturen}
Man kann Speicher schon im Code definieren oder wenn benötigt wärend der Laufzeit eines Programmes Dynamisch allozieren.

\textbf{pro Memoria : Variablen}

\begin{itemize}
	\item erleichtern u.a. den Zugriff auf Speicherstellen (anstelle Adressen)  
	\item Müssen zur Entwicklungszeit im Code definiert werden
	\item Der Speicher einer Variable wird automatisch freigegeben, sobald die Variable nicht mehr gültig ist.
	
\end{itemize}

\textbf{Dynamische Speicherverwaltung}
\begin{itemize}
	\item Speicher kann zur Laufzeit (dynamisch) vom System angefordert (alloziert) werden 
	\subitem Operator: \texttt{new} (in C: Funktion \texttt{malloc()}).
	\item Dynamisch allozierter Speicher muss wieder explizit freigegeben werden
	\subitem Operator: \texttt{delete} (in C: Funktion \texttt{free()}).
	\item Dynamischer Speicher wird nicht auf dem Stack angelegt, sondern auf dem \textbf{Heap}.
	\item Auf Dynamisch allozierter Speicher kann \textbf{nur} über Pointer zugegriffen werden.
\end{itemize}

\begin{lstlisting}[mathescape]
int* pInt = new int; // Speicher für int alloziert
char* pCh1 = new char; // Speicher für char alloziert
char* pCh2 = new char;
char* pArr = new int[100]; // Array
*pInt = 23;
std::cin >> *pCh1;
pCh2 = pCh1;
// pCh2 zeigt nun auch auf die gleiche Speicherstelle wie pCh1. Damit geht aber der Zugriff auf die Speicherstelle verloren, auf die pCh2 gezeigt hat (memory leak!)
delete pInt; // Speicher freigeben
delete pCh1; // Speicher freigeben
delete pCh2; // ergibt Fehler
pInt[22] = -45; // Wert zuweisen
delete pInt; // Fehler: nur pInt[0] wird freigegeben
delete[] pInt; // korrekter Befehl
\end{lstlisting}
Beim Anwenden des \texttt{delete}-Operator auf einen bereits freigegebenen Speicherbereich, kann Probleme verursachen. Oft wird deshalb ein Pointer nach der \texttt{delete}-Operation auf 0 (bzw. \texttt{nullptr}) gesetzt.



















	
	\newpage
	\subsection{Pointer und Referenzen als Rückgabewert und Parameterübergabe}
  		Bei Variablenübergabe (call by value) werden Kopien übergeben, welche nicht verändert werden können.\\
  		Bei Referenzübergabe (call by reference) kann die Subroutine die Werte bleibend verändern. \\
  		\textbf{Objekte einer Klasse und Strukturvariablen sollen immer by reference übergeben werden!} \\
  		

\subsection{call by reference}
  \begin{minipage}[t]{5cm}
  \begin{lstlisting}
void swap(int& a, int& b)
{
int tmp = a;
a = b;
b = tmp;
}
int main()
{
int x = 4;
int y = 3;
swap(x, y); // OK!
return 0;
}	
\end{lstlisting}
\end{minipage}
\hspace*{0.5cm}
\begin{minipage}[t]{5cm}

\begin{lstlisting}
void swap(int* a, int* b)
{
int tmp = *a;
*a = *b;
*b = tmp;
}
int main()
{
int x = 4;
int y = 3;
swap(&x, &y); // OK!
return 0;
}
\end{lstlisting}
\end{minipage}
%\hspace*{0.5cm}
\begin{minipage}[t]{4.9 cm}
\subsection{call by value}
\begin{lstlisting}
void swap(int a, int b)
{
int tmp = a;
a = b;
b = tmp;
}
int main()
{
int x = 4;
int y = 3;
swap(x, y); // keine Ausw.
return 0;
}	
\end{lstlisting}
\end{minipage}

bkljdfbkjsdfb  asfoih sdafojps dfsdfjksdh fdfks dsdf sdf
sdf sdfjoisjdf sd f
sdfj sodp f 

df ihoüsdfjs
df



	\section{Funktionen}
Funktionen sind Unterprogramme, die häufig verwendeten Code enthalten.\\
Ein Beispiel:
\begin{lstlisting}
	int add (int a, int b); 	//Prototyp
	
	//PRE: a, b > 0
	//POST: true, wenn eine das doppelte der anderen ist
	bool timestwo (int a, int b){
		bool c=false;
		return a==add(b, b) || b==add(a,a);
	}
\end{lstlisting}
Rückgabewert ist immer genau \textbf{ein} Variabeltyp (Workaround: Structs). Ohne Rückgabewert schreibt man \texttt{void}.
\subsection{Aufbau}
\texttt{rückgabewert} \texttt{funktionsname} \texttt{(argument)}\texttt{\{}

\hspace{10pt}\texttt{funktionskörper}

\hspace{10pt}\texttt{return ;}\\
\texttt{\}}
\subsection{Pre- und postconditions}
Preconditions beschreiben den Input der Funktion, Postcondition den Output und die Wirkung der Funktion. Preconditions prüft man mit\\ \texttt{assert (a>0 \&\& b>0)}
\subsection{Prototyp und Gültigkeitsberieche}
Falls eine Funktion \texttt{g}, die Funktion \texttt{f} benötigt muss diese vorab definiert sein, da sich der Gültigkeitsbereich einer Funktion nur unterhalb seiner Defintion befindet. Die formalen Argumente verhalten sich wie Variabeln und haben nur einen Lokalen Gültigkeitsbereich im Funktionsblock.
\subsection{Rekursion}
Wenn eine Funktion sich selber wieder auruft, nennt man das Rekursion. Dabei muss es eine Abbruchbedingung geben, die auch erreicht wird. Dann wird von innen aufgelöst.
\begin{lstlisting}
	int fak (int n){
		if(n==1) return 1;
		return n* fak(n-1);
	}
\end{lstlisting}




	\section{Pointer und Referenzen} 
Bei Variablenübergabe (call by value) werden Kopien übergeben, welche nicht verändert werden können.\\
Bei Referenzübergabe (call by reference) kann die Subroutine die Werte bleibend verändern. \\
\textbf{Objekte einer Klasse und Strukturvariablen sollen immer by reference übergeben werden!} \\
\subsection{call by reference}
\vspace{-13pt}
\begin{multicols}{2}
	statisch:
	\begin{lstlisting}
		void swap(int& a, int& b){
			int tmp = a;
			a = b;
			b = tmp;
		}
		int main(){
			int x = 4;
			int y = 3;
			swap(x, y);// OK!
			return 0;
		}	
	\end{lstlisting}
	dynamisch:
	\begin{lstlisting}
		void swap(int* a, int* b){
			int tmp = *a;
			*a = *b;
			*b = tmp;
		}
		int main(){
			int x = 4;
			int y = 3;
			swap(&x, &y);// OK!
			return 0;
		}
	\end{lstlisting}
\end{multicols}
\subsection{call by value}
	\begin{lstlisting}
		void swap(int a, int b){
			int tmp = a;
			a = b;
			b = tmp;
		}
		int main(){
			int x = 4;
			int y = 3;
			swap(x, y); // keine Auswirkung
			return 0;
		}	
	\end{lstlisting}
\subsection{return by reference}
\begin{lstlisting}
	int& inc(int& i){
		return ++i;
	}	
\end{lstlisting}
Der Funktionsaufruf ist nun selbst ein L-Wert, was nun Ausdrücke wie \texttt{inc(inc(x))} oder \texttt{++inc(x)} erlaubt. \textbf{Achtung} Gültigkeitsbereiche: Return by reference auf lokale Variable ist undefined behavior.
\subsection{Pointer}
Ein Ausdruck vom Typ \texttt{T*} heisst Zeiger (Auf T). Der Wert eines Zeigers ist die Adresse, auf die er zeigt. Man kommt auf die Adresse eines Objektes entweder direkt beim Erzeugen mittels \texttt{new} oder mit dem Adress-Operator \texttt{\&}.Um undefiniertes Verhalten bei der Initialisierung zu vermeinden, erstellt man einen Null-Zeiger.
\begin{lstlisting}
	int* r; //undefined behaviour
	int* q=nullptr; //(zeigt explizit ins nichts)
	
	int i = 5;
	int* p = &i; //Zeiger auf i
	std::cout<<*p; // Output: 5
\end{lstlisting}
\includegraphics[width=0.24 \textwidth]{sections/pointer}
Um nun der Wert bei der Adresse des Pointers zu erhalten braucht man den Deferenz-Operator \texttt{*}.
\subsubsection{Pointerarithmetik}
Mit Pointer-





	\section{Vektoren} 
Vektoren dienen zum Speichern gleichartiger Daten.
\subsection{Initialisierung}
\begin{lstlisting}
	std::vector<int> vec(3);
	//{0, 0, 0}
	std::vector<int> vec(4, 2);
	//{2, 2, 2, 2}
	std::vector<int> vec{4,3,2,1};
	//{4, 3, 2, 1}
	std::vector<int> vec;
	//leerer Vektor
\end{lstlisting}
\subsection{Zugriff}
Das erste Element eines Vekotrs hat index 0. Ein Zugriff auf Elemente ausserhalb der gültigen Grenze führt zu undefinierten Verhalten. C++ bietet eine optionale überprüfung.
\begin{lstlisting}
	std::vector<int> vec(3);
	vec.at(3) = 1;	//Error compiler
	vec[3] = 1; //undefined behaviour
\end{lstlisting}
\subsection{Befehle}
\begin{lstlisting}
	std::vector<int> vec{0,1};
	vec.size(); //Länge des Vektors: 2
	vec.push_back(3) //hängt wert an: {0,1,3}
	vec.clear(); //löscht Inhalt : {0,0,0}
	vec.resize(2); //ändert Grösse: {0,0}
\end{lstlisting}






	\section{Strings} 
Strings sind Arrys/Vektoren vom typ char. Mit Strings speichert man folglich längere Zeichenketten und benötigt \texttt{\#include<string>}. Dank überladener Operatoren haben Strings einige Zusatzfunktionen zu Vektoren.

\subsection{Initialisierung / Funktionen}
\begin{lstlisting}
	std::string text(3, 'u'); 	// {u, u, u}
	std::string name = "Cedric";
	name += " Renda";
	std::cout<< name = "Robin von Reding"; //false
	std::cout<< name = "Cedric Renda"; //true
\end{lstlisting}
\subsection{ASCII-Tabelle}
Werte vom Typ \texttt{int} und \texttt{char} lassen sich einfach konvertieren.
\begin{lstlisting}
	int i=97;
	char c=i;
	std::cout<<c; //Output: a
	c = 'A';
	i = c;
	std::cout<<i; //Output: A
\end{lstlisting}
Der Compiler geht dabei nach folgender Tabelle vor.
\includegraphics[width=0.24 \textwidth]{sections/ASCII-Tabelle}
\begin{tabular}{rlcrl}
	00-31:& NUL, ... & \quad & 32:& SPACE\\
	48-57:& 0-9& \quad &	65-90:& A-Z \\
	97-122:& a-z & \quad & 127:& DEL\\
\end{tabular}





	\section{Klassen}
Eine Klasse ist eine Datenstruktur wie auch Structs. Eine Klasse hat jedoch unterschiedliche Zugriffsrechte:





	\section{Container und Iteratoren}
Container sind Datenstrukturen mit einer Ansammlungen von Elementen, auf welchen Operationen ausgeführt werden können (z.Bsp Vektor).
Die Standartbiblipthek von C++ enthält diverse Container mit unterschiedlichen Eigenschaften. \todo{Beispiele Container Eigene Section???}
Um Unterschiedliche funktionen mit Container zu realisieren (z.bsp. um diese auszugeben) sind Iterator hilfreich. Jeder C++-Container implementiert seinen eigen Iterator. Gegeben sei ein Container \texttt{c}.
\begin{tabular}{p{0.08 \textwidth}|p{0.14\textwidth}}
	\texttt{it=c.begin()} & Itarator aufs erste Element\\
	\texttt{it=c.end()} & Iterator hinters letzte Element\\
	\texttt{*it} & Zugriff aufs aktuelle Element\\
	\texttt{++it} & Iterator um ein Element verschieben\\
	\texttt{it2!=it} & (oder \texttt{==}) vergleichen von Iteratoren
\end{tabular}
Iteratoren sind eine Art Containerspezifische Zeiger. Vorteil: Nutzer müssen genaue Implementierung nicht kennen.
\begin{lstlisting}
	void print(std::vector<int> vec){
	for(std::vector<int>::iterator it= vec.begin();
		it<vec.end();
		++it){
		std::cout<<*it<<" ";
		}
	}
\end{lstlisting}
Um einen solchen Iterator zu schreiben, muss ein Klasse mit den obigen Funktionen geschreiben werden. \texttt{iterator} ist eine innere Klasse von \texttt{Container}.
\begin{lstlisting}
	class Container {
		...
		public:
		class iterator {
			...
		};
		...
	};
\end{lstlisting}
Jeder Container sollte auch ein \texttt{const\_iterator} bereitstellen. Dieser wird gebraucht, wenn nur lesezugriff gestattet ist oder das Objekt selbst \texttt{const.} ist.\\
Folgende Standard librarys gehören zu den sequenziellen container welche die daten sequentiell zugreifen.

\begin{multicols}{2}
	\textbf{Sequenzielle:}
	\begin{itemize}
		\item array
		\item vector
		\item deque
		\item forward\_list
		\item list
	\end{itemize}
	
	\textbf{Adaptors:}
	\begin{itemize}
		\item stack
		\item queue
		\item priority\_queue
	\end{itemize}
	
\end{multicols}
Container adaptoren stellen ein anderes interface für sequentielle container zu verfügung


Die Assoziativen container sind sortiert implementiert und können mit der Komplexität $O(log(n))$ durchsucht werden.
\begin{multicols}{2}
	\textbf{Assoziativ:}
	\begin{itemize}
		\item set
		\item map
		\item multiset
		\item multimap
	\end{itemize}
	
	\textbf{Assoziativ unsortiert:}
	\begin{itemize}
		\item unordered\_set
		\item unordered\_map
		\item unordered\_multiset
		\item unordered\_multimap
	\end{itemize}
\end{multicols}
Die unsortierten assoziativen container sind unsortierte (hashed) Daten Strukture welche schnell $O(1)$ (ideal) oder worst-case $O(n)$ durchsucht werden können.


	\section{Subtyping, Polymorphie und Vererbung}
\subsection{Konzepte}
\subsubsection{Subtyping}
\begin{multicols}{2}
	Subtyping beschreibt das Konzept einer Typhirarchie. \texttt{Exp} repräsentiert Algemeine Ausdrücke. Jedes \texttt{Literal}, aber auch jede \texttt{Addition} ist ein \texttt{Exp}.\\ \includegraphics[width=0.12 \textwidth]{sections/subtyping}
\end{multicols}
Überall wo ein \texttt{Exp} erwartet wird, kann auch \texttt{Literal} oder \texttt{Times} genutzt werden.
\subsubsection{Polymorphie und dynamische Bindung}
Eine Variable vom statiscen Typ \texttt{Exp} kann Ausdrücke mit unterschiedlichen dynamischen Typen ''beherbergen''. Ausgeführt werden die Memeberfunktionen des dynamischen Typs.
\begin{lstlisting}
	Exp* e = new Literal(2);
	std::cout<< e->eval(); // 2
	
	e = new Addition(e,e);
	std::cout<< e->eval(); // 4 (eval von Addtion)
\end{lstlisting}
\subsubsection{Vererbung}
Manche Funktionalitäten sind für mehrere Mitglider der Typhirarchie gleich. Zum Codeduplikation zu verhindern die Funktion an Subtyp vererben.
\begin{lstlisting}
	class Exp{...};
	class BinExp : public Exp{...};
	class Times : public BinExp{...};
\end{lstlisting}
\texttt{BinExp} ist eine Subklasse von \texttt{Exp} und eine Superklasse von \texttt{Times}.
\subsection{Anwendung}
\begin{lstlisting}
	class Exp{
		public:
		virutal int size() const = 0;
		virutal double eval() const = 0;
	};
	class Literal : public Exp{
		double val;
		public:
		Literal(double v);
		int size() const;
		double eval() const;
	};
\end{lstlisting}
\texttt{Exp} ist ein abstrakte Klasse. \texttt{=0} erzwingt die Implementierung in der Subklasse. \texttt{virtual} aktiviert die Dynamische Bindung.\\
\texttt{Literal} erbt durch \texttt{public Exp} von \texttt{Exp} ist sonst aber eine normale Klasse.
Bei virtuellen Memberfunktionen bestimmt der dynamische Typ die auszuführende Memberfunktion (dynamische Bindung)  ohne virtual der statische Typ.
Gemeinsamkeiten von \texttt{Times} und \texttt{Addtion} können in \texttt{BinExp} ausgelagert werden.
\begin{lstlisting}
	class BinExp : public Exp{
		Exp* left;
		Exp* right;
		public:
		BinExp(Exp* l, Exp* r);
		int size() const;
	};
	BinExp::BinExp(Exp* l, Exp* r) : left(l), right(r) {}
	int BinExp::size() const {
		return 1 + this->left->size() + this->right->size();
	}
\end{lstlisting}
\texttt{BinExp} implementiert \texttt{eval()} nicht und ist darum auch eine abstrakte Klasse. Die Gemeinsamkeiten von \texttt{BinExp} werden weriterverwerbt.
\begin{lstlisting}
	class Addition : public BinExp {
		public:
		Addition(Exp* l, Exp* r);
		double eval() const;
	};
	Addition::Addition(Exp* l, Exp* r) : BinExp(l,r){}
	double Addition::eval() const{
		return this->left->eval() + this->left->eval();
	}
\end{lstlisting}
\texttt{Addition} erbt die Membervariabeln \texttt{left}, \texttt{right} und die Funktion \texttt{size} von \texttt{BinExp}. Genau genommen hat \texttt{Addition} jedoch kein Zugriff auf \texttt{left} und \texttt{right}, da diese \texttt{privat} sind in \texttt{BinExp}. \texttt{BinExp} bräuchte eine Memberfunktion, die \texttt{left} und \texttt{right} zurückgibt und somit den Zugriff gewährt. \todo{Das ist schon so oder???}





\end{multicols*}		
\newpage
\listoftodos[Notes]
\end{document}
