\section{Binarytrees}
Eine Liste mit jeweils 2 Nachfolgern. Dem obersten Knoten sagt man Wurzel (\textbf{root}). 
Alle Knoten (\textbf{node}) die am Ende des Baums hängen werden Blatt (\textbf{leaf}) genannt. 
Höhe eines Baums = maximale Anzahl Knoten zwischen root und leaf.
Folgendes Element links: kleiner als Knoten, rechts grösser.
\begin{lstlisting}[mathescape]
struct tNode {
	int key;
	tNode *left, *right;
};
\end{lstlisting}

Füge neuen Knoten mit Key k ein. Rekursive Methode.
\begin{lstlisting}[mathescape]
void insert(tNode *p, int k){
if(p==0{
p = new tNode;
p->key = k;
p->left = NULL; p->right = NULL;}
else if(p->key > k) insert(p->left,k);
else insert(p->right, k);}
// Aufruf:
tNode *root = NULL; insert(root,2); 
\end{lstlisting}
Iterative Methode. Rückgabewert ist ein Pointer.
\begin{lstlisting}[mathescape]
tNode* search(tNode *root, int k){
tNode *p = root;
while(p){
if(p->key == k) return p;
if(p->key > k) p = p->left;
else p = p->right;}}
//Aufruf:
tNode *x = search(root,2);
\end{lstlisting}
\todo{Binary Trees fertig machen und noch ein Beispiel dazu}
