\section{Variablen}
Wertebereich des Integers ist mit $n$-bits für \texttt{signed} $-2^{n-1}$ to $2^{n-1}-1$ und für \texttt{unsigned} $0$ to $2^{n}-1$.
\begin{center}
	\begin{tabular}{ | m{2.4cm} | m{0.6cm}| m{2.6cm} | } 
		\hline
		Type & Bytes & Range\\
		\hline\hline
		\texttt{char} & 1 & $-128$ to $127$ \\ 
		\texttt{unsigned char} & 1 & $0$ to $255$ \\ 
		\texttt{WCHAR\_T} & 4 & 1 Wide Char \\ 
		\hline
		\texttt{unsigned int} & 4 & $0$ to $ 4,294,967,295$ \\ 
		\texttt{int} & 4 & $-2,147,483,648$ to $2,147,483,647$ \\ 
		\texttt{unsigned short int} & 2 & $0$ to $65,535$ \\
		\texttt{short int} & 2 & $-32,768$ to $32,767$ \\  
		\texttt{unsigned long int} & 8 & $0$ to $2^{64}-1$ \\ 
		\texttt{long int} & 8 & $-2^{63}$ to $2^{63}-1$ \\
		\hline
		\texttt{float} & 4 & - \\ 
		\texttt{double} & 8 & - \\
		\texttt{long double} &16 & - \\
		\hline
	\end{tabular}
\end{center}


\subsection{Variablennamen}
Keine Leerzeichen, Satzzeichen oder \_ Symbole Keine Zahl oder am Anfang case sensitivity – Gross - Kleinschreibung beachten

\subsection{Einfache Variablen deklarieren}
\begin{lstlisting}
int a,a2;   int b (1);
int b = 10;  int b {1};
float c = a*b - 0.5;
\end{lstlisting}
\subsection{Type Casts}
Änderung einer Variable in einen anderen Type
\begin{lstlisting}	
double a = 1.5; int b;
b = int (a);
b = (int) a; // b=1
7/2 = 3 , 7/(double)2 = 7/2.0 = 3.5
double(7/2) = 3.0 , int(19/10.0) = 1
\end{lstlisting}
\textbf{const\_cast}
Ausschliesslich bei der Entfernung des const-Qualifkators. Syntax: \texttt{const\_char<Zieltyp>expression}
\begin{lstlisting}	
const char* findSubString(const char* str, const char* subStr)
{
return strstr(const_cast<char*>str, const_cast<char*>subStr);
}
\end{lstlisting}

\textbf{static\_cast}
Umwandeln eines Klassenobjekt in ein Objekt seiner Basisklasse. Syntax ist analog zu \texttt{const\_cast}.
\begin{lstlisting}	
static_cast<SuperHero>Batman;
\end{lstlisting}


\textbf{dynamic\_cast}
Umwandlung von Polymorphen Objekten im Zusammenhang mit dem Typsystem von C++.
\begin{lstlisting}	
dynamic_cast<SuperHero*>p;
\end{lstlisting}


\textbf{reinterpret\_cast}
Neue Interpretation der zugrunde liegenden Bitkette.
\begin{lstlisting}	
reinterpret_cast<int*>str // char Pointer str wird in int-Pointer gewandelt.
\end{lstlisting}


\subsection{Enum}
Enum ist ein Aufzählungstyp. Die Konstanten aus der Enum
kann man im Programm verwenden.
\begin{lstlisting}	
enum farbe {ROT, BLAU, GELB};
farbe f = ROT;
if(f != BLAU) { }; 
\end{lstlisting}

\subsection{Hexadezimaler Code \& Adressen}
0,1,...,9,A,B,C,D,E,F (hex) anstelle von 0,1,...,14,15,16 (dec)
Adressen werden hexadezimal angegeben. $a,a+1,a+2,a+3,...$

\begin{center}
	\begin{tabular}{ ll } 
		\hline
int,float(4byte) & double (8byte)\\
\hline
0x22ff70 & 0x22ff70\\
0x22ff74 & 0x22ff78\\
0x22ff78 & 0x22ff80\\
		\hline
	\end{tabular}
\end{center}

\todo{Schönners Beispiel und bessere Formatierung}\subsection{Fliesskommazahlen}
Fliesskommazahlen sind durch 4 Zhalen definiert:
\begin{itemize}
	\item Die Basis $\beta \geqslant 2$
	\item Die Präzision $p \geqslant 1$
	\item Der kleinste Exponent $e_{min}$
	\item Der grösste Exponent	$e_{max}$
\end{itemize}
Fliesskommazahlen Tabu!
\begin{itemize}
	\item Test keine gerundeten Fliesskommazahlen auf Gleichheit! z.b in for-loop
	\item Addiere keine zwei Zahlen  sehr unterschiedlicher Grösse! 
	\item Subtrahiere keine zwei Zahlen sehr ähndlicher Grösse! Auslöschung!

\end{itemize}
$F$ enthält $\displaystyle \pm\sum_{i=0}^{p-1}d_i \beta^{-i}\cdot \beta^{e} =  \pm d_{0\text{\textbullet}} d_1\dots d_{p-1} \times \beta^e$
In der normalisierten Darstellung $F^{\ast}$ gilt $d_0 \neq 0$, $F^{\ast}(\beta,p,e_{min},e_{max})$ 

\textbf{Bsp:$F^{\ast}({\color{brown}{2}},{\color{red}{3}},{\color{blue}{-2}},{\color{cyan}{2}})$} Vorzeichen ist in Precision
\begin{center}
	\begin{tabular} { m{0.8cm} | m{0.8cm} m{0.95cm} m{0.7cm} m{0.7cm}m{0.6cm} }
	 $d_{0\text{\textbullet}} d_1 d_2$ & ${\color{blue}{e=-2}}$&$e=-1$&$e=0$&$e=1$&${\color{cyan}{e=2}}$\\
	 \hline
	 ${\color{red}{1.00}}{\color{brown}{_2}}$ & 0.25   & 0.5   & 1    & 2   & 4\\
	 ${\color{red}{1.01}}{\color{brown}{_2}}$ & 0.3125 & 0.625 & 1.25 & 2.5 & 5\\
	 ${\color{red}{1.10}}{\color{brown}{_2}}$ & 0.375  & 0.75  & 1.5  & 3   & 6\\
	 ${\color{red}{1.11}}{\color{brown}{_2}}$ & 0.4375 & 0.875 & 1.75 & 3.5 & 7\\
	 \hline
	\end{tabular}
\end{center}
Anz. (positive) Zahlen: $\beta^{p-1}\cdot (e_{max}-e_{min}+1) $
\begin{center}
	\begin{tabular}{ ll } 
		float&1b$\rightarrow$sign, 8b$\rightarrow$exp, 23b$\rightarrow$mantisse\\
		&Wert = $(-1)^{\texttt{S}} \cdot 2^{(\texttt{E-127})} \cdot (1.\texttt{F})$\\
		&Bsp: $0.125 = 2^{3} \Rightarrow \texttt{S} \rightarrow 0, \texttt{E} \rightarrow 124, \texttt{F} \rightarrow 0$\\
		\hline
		 & 0 | 01111100 | 00000000...0 = 0.125\\
		 & 0 | 01111111 | 00000000...0 = 1\\
		 & 1 | 01111111 | 11000000...0 = -1.75\\
		 & 0 | 00000000 | 00000000...0 = 0\\
		 & 0 | 11111111 | 00000000...0 = +infty\\
		 & 0 | 00000000 | 00000000...0 = NaN\\
		\hline
		double& 1b$\rightarrow$sign, 11b$\rightarrow$exp, 52b$\rightarrow$mantisse\\
		& Wert = $(-1)^{\texttt{S}} \cdot 2^{(\texttt{E-1023})} \cdot (1.\texttt{F})$\\
	\end{tabular}
\end{center}















