\section{Library}
\subsection{Vector}
\rowcolors{2}{gray!20}{white}
\begin{center}
	\renewcommand\tabularxcolumn[1]{m{#1}}
	\begin{tabularx}{\columnwidth}{ | l  X | } 
		\hline
		v.at(pos)&access specified element with bounds checking\\
		
		v[pos] &access specified element\\
		
		v.front() &Reference to the first element\\
		
		v.back() & Reference to the last element\\
		
		v.data() & direct access to the underlying array\\
		
		v.begin()& returns an iterator to the beginning\\
		
		v.end() & returns an iterator to the end\\
		
		v.rbegin()& returns a reverse iterator to the beginning\\
		
		v.rend()& returns a reverse iterator to the end\\
		
		v.empty()& checks whether the container is empty\\
		
		v.size() & returns the number of elements\\
		
		v.max\_size()& returns the maximum possible number of elements\\
		
		v.reserve(new\_cap)& new storage is allocated\\
		
		v.capacity()& returns the number of elements that can be held in currently allocated storage\\
		
		v.shrink\_to\_fit() & reduces memory usage by freeing unused memory\\
		
		v.clear()& clears the contents\\
		
		v.insert(pos, val) &\\
		
		v.emplace(pos, args) & constructs element in-place\\ 
		
		v.erase(pos) & erases elements. v.erase(v.begin()) first element\\
	
		v.push\_back(arg) & adds an element to the end\\
	
		v.emplace\_back() & 	constructs an element in-place at the end\\
		
		v.pop\_back() & removes the last element\\
		
		v.resize(arg) & changes the number of elements stored, removes from back if smaller\\
				
		v1.swap(v2) & swaps the contents \\
		std::swap(v1, v2) & swaps the contents \\
		\hline
	\end{tabularx}
\end{center}





\subsection{cmath}
Some important cmath functions
\rowcolors{2}{gray!20}{white}
\begin{center}
	\renewcommand\tabularxcolumn[1]{m{#1}}
	\begin{tabularx}{\columnwidth}{ | l  X | } 
		\hline
		\texttt{cos(param)}		&Compute cosine \\
		\texttt{sin(param)}		&Compute sine \\
		\texttt{tan(param)}		&Compute tangent \\
		\texttt{exp(param)}		&Compute exponential function\\
		\texttt{log(param)}		&Compute natural logarithm \\
		\texttt{log10(param)}	&Compute common logarithm \\
		\texttt{modf(param, \&intpart)} & Break into fractional and integral parts\\
		\texttt{exp2(param)} &Compute binary exponential function \\
		\texttt{pow(base, exp)} &Raise to power $b^{exp}$	\\
		\texttt{sqrt(param)} &Compute square root 	\\
		\texttt{cbrt(param)} &Compute cubic root 	\\
		\texttt{hypot(x,y)} & The square root of $(x^2+y^2)$.	\\
		\texttt{ceil(param)} & Round up value	\\
		\texttt{floor} &Round down value	\\
		\texttt{fmod(num, denom)} &	Compute remainder of division\\
		\texttt{trunc(param)} &Truncate value	\\
		\texttt{round(param)} &	Round to nearest \\
		\texttt{rint(param)} &Round to integral value	\\
		\texttt{nearbyint(param)} &Round to nearby integral value	\\
		\texttt{fabs(param)} & Absolute value of floating numbers	\\
		\texttt{abs(param)} &Compute absolute value \\
		\texttt{fma(x,y,z)} &The result of $x\cdot y+z$	\\
		\hline
	\end{tabularx}
\end{center}