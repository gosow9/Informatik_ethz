\section{Structures}

Structs werden vor der main() Funktion definiert.
\begin{lstlisting}[mathescape]
struct point {
int x, y;
double gamma;
}p,q;   // p,q schon definiert

point k;   // Neuer "point" definieren
k.x = 2;  // Variable in struct definieren
point* ptr = &p;
k = {1,2,0.75}; // schnell initialisieren
p={};// Alle Elemente auf 0 initialisiert
q ={1,2,3}; // rest wird mit 0 aufgefüllt
p = q; // ist gleich wie
p.x = q.x; 
p.y = q.y;
p.gamma = q.gamma;
cout << ptr->gamma<<p.y<<q.x; //Output: 321
\end{lstlisting}

\textbf{Falsch Rekursion:} 
\begin{lstlisting}[mathescape]
struct point {int x; point y;}; // Keine Rekursion
\end{lstlisting}
\textbf{Richtig Selbstreferenzierung:} 
\begin{lstlisting}[mathescape]
struct node
{
int data;
struct node *next; // <-self reference
};
\end{lstlisting}
\textbf{Strichpunkt am Ende nicht vergessen} 
\texttt{struct point \{int i; double y;\};}

\subsection{Funktionen in Structs}
Die Konstruktor-Funktion wird bei der Generierung eines neuen Structs aufgerufen.
\begin{lstlisting}[mathescape]
struct Bar
{
	Bar() {//Konstruktor}
};
\end{lstlisting}
Funktionen können auch ausgelagert werden:
\begin{lstlisting}[mathescape]
struct Bar
{
void bier();
}bqm;
void Bar::bier() {};
\end{lstlisting}
Aufrufen der Funktion: \texttt{bqm.bier();}


























