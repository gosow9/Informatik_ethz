\section{EBNF}
Erweiterte Backus-Naur-Form 
\begin{center}
	\begin{tabular}{|l|c|}
		\hline
\textbf{Verwendung} & \textbf{Zeichen} \\
\hline
Definition & =\\\hline
Aufzählung & ,\\\hline
Endezeichen & ;\\\hline
Alternative & |\\\hline
Option & [...]\\\hline
Optionale Wiederholung & $\left\lbrace ...\right\rbrace $ \\\hline
Gruppierung & (...)\\\hline
Anführungszeichen 1. Variante & "..."\\\hline
Anführungszeichen 2. Variante & '...'\\\hline
Komentar & (*...*)\\\hline
Spezielle Sequenz & ?...?\\\hline
Ausnahme & -\\
\hline
	\end{tabular}
\end{center}
Beispiel für einen Rechner.
Eine Zahl ist eine Folge von Ziffern. Eine Folge von Ziffern ist eine Ziffer oder eine Ziffer gefolgt von Ziffern.
\begin{lstlisting}[mathescape]
number = digits
digit = '0'|'1'|'2'|'3'|'4'|'5'|'6'|'7'|'8'|'9'
digits = digit | digit digits.

\end{lstlisting}
