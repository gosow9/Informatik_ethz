\section{Klassen}
Eine Klasse ist eine Datenstruktur wie auch Structs. Eine Klasse hat jedoch unterschiedliche Zugriffsrechte auf die internen Variabeln, Objekte genannt.
\begin{itemize}
	\item[\texttt{private}] - (default) Elemente, meist interne Variablen, können nur innerhalb der Klasse angesprochen werden. Memeberfunktionen(Methoden) können darauf zugreifen. 
	\item[\texttt{public}] - Elemente, die meisten Methoden, können von innerhalb und ausserhalb der Klasse angesprochen werden.
	\item[\texttt{protected}] - Elemente können von innerhalb der Klasse und von abgeleiteten Klassen angesprochen werden.
\end{itemize}
\begin{lstlisting}
	class klassenname{
		private:
			int n;					//Membervariabeln
			int d;
		public:
			Klassenname(int, int);	//Konstruktor
			int f1(int);			//Memeberfunktionen (prototyp)
			void f2();
	};
\end{lstlisting}
\subsection{Konstruktoren}
Konstruktoren sind spezielle Memeberfunktionen, die den Namen der Klasse tragen. Sie können auch überladen werden und werden bei der Variabelndeklaration aufgerufen.
\begin{lstlisting}
	class rational{
		...
		public:
		rational (int num, int den);
		rational ();
	};
	rational::rational (int num, int den)
	: n(num), d(den){ //Variabeln initialisierung
		assert(den!=0); // Funktionsrumpf
	}
	rational::rational () : n(0), d(1){};
\end{lstlisting}
Damit eine Variable bereits definiert werden kann ohne ''initialisiert'' zu werden, sollte ein Default-Konstruktor erstellt werden. Dieser enthält keine Argumente setzt die Membervariablen auf ein spezifischen default-Wert. Nun kann ein Objekt vom Typ rational unterschiedlich initialisiert werden. 
\begin{lstlisting}
	rational r
	rational r1(1,2);
	rational r2 = rational(1,2);
\end{lstlisting}
Der \texttt{this->}-Pointer ist ein Pointer auf das Aktuelle Objekt.



