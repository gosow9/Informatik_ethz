\section{Klassen}
Eine Klasse ist eine Datenstruktur wie auch Structs. Eine Klasse hat jedoch unterschiedliche Zugriffsrechte auf die internen Variabeln, Objekte genannt.
\begin{itemize}
	\item[\texttt{private}] - (default) Elemente, meist interne Variablen, können nur innerhalb der Klasse angesprochen werden. Memeberfunktionen(Methoden) können darauf zugreifen. 
	\item[\texttt{public}] - Elemente, die meisten Methoden, können von innerhalb und ausserhalb der Klasse angesprochen werden.
	\item[\texttt{protected}] - Elemente können von innerhalb der Klasse und von abgeleiteten Klassen angesprochen werden.
\end{itemize}
\begin{lstlisting}
	class stack{
		private:
			int height;			//Membervariabeln
			node* topn;					
		public:
			stack(node* t);		//Konstruktor
			void push(int data):			//Memeberfunktionen (prototyp)
			void pop();
			void top() const;
	}; 		//Strichpunkt nicht vergessen
\end{lstlisting}
\todo{Beispiel, dass alle Themen der Klasse abdeckt}
\subsection{Konstruktoren}
Konstruktoren sind spezielle Memeberfunktionen, die den Namen der Klasse tragen. Sie können auch überladen werden und werden bei der Variabelndeklaration aufgerufen.
\begin{lstlisting}
	class stack{
		...
	};
	stack::stack (node* t)
	: topn(t), height(1){ //Variabeln initialisierung
		assert(den!=0); // Funktionsrumpf
	}
	rational::rational () : n(0), d(1){};
\end{lstlisting}
Damit eine Variable bereits definiert werden kann ohne ''initialisiert'' zu werden, sollte ein Default-Konstruktor erstellt werden. Dieser enthält keine Argumente setzt die Membervariablen auf ein spezifischen default-Wert. Nun kann ein Objekt vom Typ rational unterschiedlich initialisiert werden. 
\begin{lstlisting}
	rational r
	rational r1(1,2);
	rational r2 = rational(1,2);
\end{lstlisting}
Der \texttt{this->}-Pointer ist ein Pointer auf das Aktuelle Objekt.
\subsection{Destruktoren}
Der Destruktor ist eine eindeutige Memberfunktion der Klasse \texttt{klassenname} mit dem Namen \texttt{~klassenname()}. Diese Funktion wird automatisch aufgerufen, wenn die Speicherdauer eines Klassenobjekts endet (z. Bsp beim Aufruf von \texttt{delete} oder wenn der Gültigkeitsbereich des Objekts endet.)
\todo{Beispiel}
\subsection{Copy-Konstruktor}
Der Copy-Konstruktor ist eine eindeutige Memberfunktion der Klasse \texttt{klassenname} mit dem Namen \texttt{klassenname(const klassenname\& x)}. Er wird automatisch aufgerufen, wenn Werte vom Typ \texttt{klassenname} mit Werten vom Typ \texttt{klassenname} initialisiert werden.
\begin{lstlisting}
	klassenname k = t //(t ist vom Typ klassenname)
	klassenname k(t);
\end{lstlisting}
Exisitiert kein Copykonstruktor, wird automatisch memberweise initialisiert (dies kann zu Fehler führen).
\todo{Beispiel}
\subsection{Zuweisungsoperator}
Achtung! Zuweisung $\neq$ Initialisierung. Für volle Funktionalität wollen wir zusätzlich noch eine Überladung vom \texttt{operator=} als Memberfunktion. Ist ähnlich wie der Copy-Konstruktor, aber ohne Initialisierung und zusätzlicher Freigabe des ''alten'' Wertes.
\begin{lstlisting}
	klassenname k, t; //(t ist vom Typ klassenname)
	...
	k=t;
\end{lstlisting}
Existiert kein Zuweisungsoperator, wird automatisch memberweise zugewiesen (dies kann zu Fehler führen).
\begin{lstlisting}
	stack& stack::operator= (const stack& s){
		if(topn != s.topn){
			stack copy = s; // Copy-Konstruktor
			std::swap(topn, copy.topn); // copy hat jetzt Müll und this die kopierten Werte
		} //copy wird aufgeräumt.
	return *this; //Rückgabe als L-Wert
	}
\end{lstlisting}




