\section{Basiswechsel}
\begin{center}
	$a_{b_1} \underrightarrow{\text{\qquad I\qquad}} a_{10} \underrightarrow{\text{\qquad II\qquad}} a_{b_2}$
\end{center}
Beispiel: ($b_1 = 2$, $b_2 = 5$, $a_{b_1} = a_2 = 0b11011$)
\begin{itemize}
	\item[I] Die Dezimalzahl ergibt sich aus
	\item[] $a_{10} = 1\cdot2^4 + 1\cdot2^3 + 0\cdot2^2 + 1\cdot2^1 + 1\cdot2^0 = 27$ 
	\item[II] Die Quinärzahl ergibt sich aus folgendem Algorithmus
	\begin{center}
		\begin{tabular}{lll}
		Koeff. 1 & Koeff. 2 & Koeff. 3\\
		$27 \% 5 = \textbf{2}$ & $5\%5 = \textbf{0}$ & $1\%5 = \textbf{1}$ \\
		$27 - 2 = 25$ & $5 - 0 = 5$ & $1 -1 = 0 $\\
		$25/5 = 5$ & $5/5 = 1$ & $0/5 = 0$ \\
	\end{tabular}
	\end{center}
$\Rightarrow a_{b_2} = a_5 = 1 0 2$
\end{itemize}
Gleicher Algorithmus für (II) anderes Beispiel:
\begin{center}
	\begin{tabular}{ lll } 
	13	&	1	& Dezimalzahl durch 2 teilen und rest\\
	6	& 	0	& notieren. Bits von unten nach oben\\
	3	&	1	& lesen.\\
	1	&	1	&6 bsp: 13 = 1101\\
	0 & & \\
	\end{tabular}
Bemerkung: $0b$ steht für Binär, $0x$ für Hexadezimal.
\end{center}























