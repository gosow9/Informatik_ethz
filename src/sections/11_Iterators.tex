\section{Container und Iteratoren}
Container sind Datenstrukturen mit einer Ansammlungen von Elementen, auf welchen Operationen ausgeführt werden können (z.Bsp Vektor).
Die Standartbiblipthek von C++ enthält diverse Container mit unterschiedlichen Eigenschaften. \todo{Beispiele Container Eigene Section???}
Um Unterschiedliche funktionen mit Container zu realisieren (z.bsp. um diese auszugeben) sind Iterator hilfreich. Jeder C++-Container implementiert seinen eigen Iterator. Gegeben sei ein Container \texttt{c}.
\begin{tabular}{p{0.08 \textwidth}|p{0.14\textwidth}}
	\texttt{it=c.begin()} & Itarator aufs erste Element\\
	\texttt{it=c.end()} & Iterator hinters letzte Element\\
	\texttt{*it} & Zugriff aufs aktuelle Element\\
	\texttt{++it} & Iterator um ein Element verschieben\\
	\texttt{it2!=it} & (oder \texttt{==}) vergleichen von Iteratoren
\end{tabular}
Iteratoren sind eine Art Containerspezifische Zeiger. Vorteil: Nutzer müssen genaue Implementierung nicht kennen.
\begin{lstlisting}
	void print(std::vector<int> vec){
	for(std::vector<int>::iterator it= vec.begin();
		it<vec.end();
		++it){
		std::cout<<*it<<" ";
		}
	}
\end{lstlisting}
Um einen solchen Iterator zu schreiben, muss ein Klasse mit den obigen Funktionen geschreiben werden. \texttt{iterator} ist eine innere Klasse von \texttt{Container}.
\begin{lstlisting}
	class Container {
		...
		public:
		class iterator {
			...
		};
		...
	};
\end{lstlisting}
Jeder Container sollte auch ein \texttt{const\_iterator} bereitstellen. Dieser wird gebraucht, wenn nur lesezugriff gestattet ist oder das Objekt selbst \texttt{const.} ist.\\
Folgende Standard librarys gehören zu den sequenziellen container welche die daten sequentiell zugreifen.

\begin{multicols}{2}
	\textbf{Sequenzielle:}
	\begin{itemize}
		\item array
		\item vector
		\item deque
		\item forward\_list
		\item list
	\end{itemize}
	
	\textbf{Adaptors:}
	\begin{itemize}
		\item stack
		\item queue
		\item priority\_queue
	\end{itemize}
	
\end{multicols}
Container adaptoren stellen ein anderes interface für sequentielle container zu verfügung


Die Assoziativen container sind sortiert implementiert und können mit der Komplexität $O(log(n))$ durchsucht werden.
\begin{multicols}{2}
	\textbf{Assoziativ:}
	\begin{itemize}
		\item set
		\item map
		\item multiset
		\item multimap
	\end{itemize}
	
	\textbf{Assoziativ unsortiert:}
	\begin{itemize}
		\item unordered\_set
		\item unordered\_map
		\item unordered\_multiset
		\item unordered\_multimap
	\end{itemize}
\end{multicols}
Die unsortierten assoziativen container sind unsortierte (hashed) Daten Strukture welche schnell $O(1)$ (ideal) oder worst-case $O(n)$ durchsucht werden können.

