%Darstellung Dokument =========================================================================================

\documentclass[10pt,twoside,a4paper,fleqn]{article}			%\documentclass[Optionen]{Dokumentklasse} => [Schriftgrösse, Layout, Papierformat, Gleichungen Linksbündig]{Art des Dokuments}
\usepackage[left=5mm,right=5mm,top=5mm,bottom=2.5mm,includeheadfoot]{geometry}	%Seitenränder festlegen


%wichtige Standard-Packages ===================================================================================

%Darstellung Text
\usepackage[utf8]{inputenc}		%LaTeX arbeitet standardmässig mit der "normalen" ASCII Codierung. Um Sonderzeichen verwenden zu können, muss deshalb explizit die UTF-8 Codierung eingestellt werden. Dazu wird das inputenc Package benötigt.
\usepackage[T1]{fontenc}		%Um die Schriftart so zu laden, dass alle Sonderzeichen nach west-europäischem Standard zur Verfügung stehen, muss die Font Encoding auf "T1" eingestellt werden:
\usepackage[ngerman]{babel}		%Das babel Paket stellt eine verbesserte Silbentrennung zur Verfügung. Wichtig ist, dass die richtige Sprache ausgewählt wird. In der Regel werden english oder ngerman (n für neue Rechtschreibung) verwendet.


%Mathematik
\usepackage{amsmath}		%Das amsmath Paket der American Mathematical Society stellt erweiterte Mathematik Zeichen, Operationen und Umgebungen zur Verfügung.
\usepackage{array}			%Definiert die array-Umgebung, die im mathematischen Modus zur Erzeugung von Matrizen dient.
\usepackage{amssymb}		%Zusätzliche Mathematische Symbole wie zum Beispiel der Implikationspfeil oder Mengen-Buchstaben, siehe auch: http://milde.users.sourceforge.net/LUCR/Math/mathpackages/amssymb-symbols.pdf

%\setlength{\mathindent}{0pt}	%Damit die Gleichungen linksbündig sind (z.B. in: /begin{equation}/begin{align}), braucht assymb
	
	
%weitere Darstellungen
\usepackage{fancybox}	%bietet mehr Möglichkeiten, wie Textboxen benutzt werden können
\usepackage{color}
\usepackage{multicol}	%Diese Umgebung erlaubt, Teile des Dokumentes mehrspaltig zu setzen.
\usepackage{verbatim}	%Macht mehrzeilige Kommentare innerhalb des Codes möglich. Bsp.: \begin{verbatim*} Mehrzeiliger Kommentar \end{verbatim*}


%Verweise
\usepackage{hyperref}	%Dieses Paket wandelt alle internen Verlinkungen in klickbare Verweise um. Dazu gehören auch die Einträge im Inhaltsverzeichnis und im Abbildungsverzeichnis. Ein Klick auf einen entsprechenden Eintrag führt somit direkt in die entsprechende Stelle im Dokument.
\usepackage{lastpage}	%Ref­er­ence the num­ber of pages in your LaTeX doc­u­ment through the in­tro­duc­tion of a new la­bel which can be ref­er­enced like \pageref{LastPage} to give a ref­er­ence to the last page of a doc­u­ment. It is par­tic­u­larly use­ful in the page footer that says: Page N of M. 	

%\usepackage[ngerman]{varioref}				%varioref: ist ein Paket, das für intelligente Querverweise sorgt. Verwendet wird es genau wie \ref, jedoch sieht das Ergebnis, je nachdem wo die Stelle auf die verwiesen wird steht, verschieden aus.  => Wird teilweise in den Zusammenfassungen gebraucht.

%PDFs einbinden
\usepackage{pdfpages}


%Kop- und Fusszeile
\usepackage{fancyhdr}	%Fancyhdr ist ein eingeführtes, eigenständiges Paket zur einfacheren Manipulation von Kopf- und Fußzeile. Es definiert Befehle für den Seitenstil fancy

%Bilder
\usepackage{graphicx}	%Das Paket graphicx ermöglicht es externe Graphiken einzubinden. Wichtigster Befehl ist dabei \includegraphics. LaTeX selbst behandelt das Bild genau wie normalen Text. 
\usepackage{wrapfig}	%Das Paket wrapfig ermöglicht es von Schrift umflossene Bilder und Tabellen einzufügen. 

%Tabellen
\usepackage{rotating}	%Package für Tabelle im Querformat

%Auflistungen
\usepackage{paralist}	%Das Paralist Paket bietet eine Erweiterung beziehungsweise eine Modifikation der bereits bestehenden Listenumgebungen an.
\usepackage{listings}



% Zusätzliche Einstellungen =============================================================================

\raggedright			% \raggedright removes paragraph indentation


%PDF Infos =============================================================================================

\hypersetup{pdfauthor={\authorinfo},pdftitle={\titleinfo},colorlinks=false} %linkbordercolor=white
\author{\authorinfo}
\title{\titleinfo}



% Layout =================================================================================================

%Kopf- und Fusszeile
\pagestyle{fancy}
\fancyhf{}


%Linien oben und unten
\renewcommand{\headrulewidth}{0.25pt} 
\renewcommand{\footrulewidth}{0.25pt}

%Kopfzeile
\fancyhead[L]{\titleinfo{ }\tiny{\textit{(\versioninfo)}}}
\fancyhead[R]{Seite \thepage { }von \pageref{LastPage}}

%Fusszeile
\fancyfoot[L]{\footnotesize{\authorinfo}}
\fancyfoot[C]{\footnotesize{\licence \quad $\rightarrow$ \href{https://github.com/HSR-Stud}{Github: HSR-Stud}}}
\fancyfoot[R]{\footnotesize{\today}}

%\fancyfoot[C]{{\includegraphics[width=1.6cm]{header/licence/cc_by-nc-sa/small.png}} 