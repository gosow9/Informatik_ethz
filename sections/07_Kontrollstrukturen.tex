\section{Kontrollstrukturen}
\textbf{if()}
\begin{lstlisting}
if(a==10){
b=15;
}
else if(a==11) b=14;
else b=10;
// oder kurz
a==10?b=15:a==11?b=14:b=10;
\end{lstlisting}

\textbf{for()}
\begin{lstlisting}
for(int i=0; i<10; i++)
{
a=a+i;
} // abbruch mit break;
\end{lstlisting}

\textbf{while()}
\begin{multicols}{2}
	\begin{lstlisting}
while(b<20){
b++;
}	
	\end{lstlisting}
oder
	\begin{lstlisting}
do {
b--; //min 1 x
}
while(b!=15);
	\end{lstlisting}
\end{multicols}
abbrechen mit break (immer nur die innere Schleife)
Überspringen des Rests des Rumpfes zur nächsten
Auswärtung mit continue;


\textbf{switch case()}
\begin{lstlisting}
switch(a) {
case 15:cout<<"a=15";break;
case 14:cout<<"a=14";break; //a==14
default:cout<<"a!=15,a!=14"; //else
}
\end{lstlisting}

\textbf{Äquivalente Strukturen}
Man kann verschiedene Strukturen verwenden um ein und dasselbe auszudrücken: 
\begin{lstlisting}
int i=0;
do {
i=i+1;
if (i==10) break;
}while(true); //aka immer
\end{lstlisting}
...ist äquivalent zu... 
\begin{lstlisting}
for(int i=0;i!=10;i++){} 
\end{lstlisting}


\textbf{Endlosschleifen}
Man kann verschiedene Strukturen verwenden um ein und dasselbe auszudrücken: 
\begin{lstlisting}
int i=10;
do {
i=i+1;
if (i==10) break;
}while(true);
for(int i=3;i!=20;i=(i+3)%300)
int i=99;
while(i>10){
i--;
if(i==15) i*=6;} 
\end{lstlisting}




