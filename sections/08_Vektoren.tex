\section{Vektoren} 
Vektoren dienen zum Speichern gleichartiger Daten.
\subsection{Initialisierung}
\begin{lstlisting}
	std::vector<int> vec(3);
	//{0, 0, 0}
	std::vector<int> vec(4, 2);
	//{2, 2, 2, 2}
	std::vector<int> vec = {4,3,2,1};
	//{4, 3, 2, 1}
	std::vector<int> vec;
	//leerer Vektor
\end{lstlisting}
\subsection{Zugriff}
Das erste Element eines Vekotrs hat index 0. Ein Zugriff auf Elemente ausserhalb der gültigen Grenze führt zu undefinierten Verhalten. C++ bietet eine optionale überprüfung.
\begin{lstlisting}
	std::vector<int> vec(3);
	vec.at(3) = 1;	//Error compiler
	vec[3] = 1; //undefined behaviour
\end{lstlisting}
\subsection{Funktionen}
Einige Funktionen aus der \texttt{vector} Bibliothek:
\begin{lstlisting}
	std::vector<int> vec{0,1};
	vec.size(); //Länge des Vektors: 2
	vec.push_back(3) //hängt wert an: {0,1,3}
	vec.pop_back(3) //hängt wert an: {0,1}
	vec.clear(); //löscht Inhalt : {0,0}
	vec.resize(3); //ändert Grösse: {0,0,0}
	vec.insert(1,3); //fügt Wert ein: {0,3,0,0}
\end{lstlisting}
\subsection{Multidimensionale Vektoren}
Eine Matrix (2. dimensionaler Vektor) ist ein Vektor, dessen Einträge Vektoren sind.
\begin{lstlisting}
	std::vector<std::vector<int>> mat = {
	{00,01,02},
	{10,11,12},
	{20,21,22}};
	std::cout<<mat[1][2]; //Output: 12
\end{lstlisting}
Praktisch sind in diesem Falle Namespaces:
\begin{lstlisting}
	using vector = std::vector<int>;
	using matrix = 	std::vector<std::vector<int>>;
\end{lstlisting}
\textbf{Wichtig}: Arrays werden fast immer per Referenz übergeben.




