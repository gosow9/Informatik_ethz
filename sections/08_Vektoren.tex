\section{Vektoren} 
Vektoren dienen zum Speichern gleichartiger Daten.
\subsection{Initialisierung}
\begin{lstlisting}
	std::vector<int> vec(3);
	//{0, 0, 0}
	std::vector<int> vec(4, 2);
	//{2, 2, 2, 2}
	std::vector<int> vec{4,3,2,1};
	//{4, 3, 2, 1}
	std::vector<int> vec;
	//leerer Vektor
\end{lstlisting}
\subsection{Zugriff}
Das erste Element eines Vekotrs hat index 0. Ein Zugriff auf Elemente ausserhalb der gültigen Grenze führt zu undefinierten Verhalten. C++ bietet eine optionale überprüfung.
\begin{lstlisting}
	std::vector<int> vec(3);
	vec.at(3) = 1;	//Error compiler
	vec[3] = 1; //undefined behaviour
\end{lstlisting}
\subsection{Befehle}
\begin{lstlisting}
	std::vector<int> vec{0,1};
	vec.size(); //Länge des Vektors: 2
	vec.push_back(3) //hängt wert an: {0,1,3}
	vec.clear(); //löscht Inhalt : {0,0,0}
	vec.resize(2); //ändert Grösse: {0,0}
\end{lstlisting}





