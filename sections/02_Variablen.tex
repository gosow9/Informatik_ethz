\section{Variablen}

\begin{center}
	\begin{tabular}{ |l|l|l| } 
		\hline
		 \texttt{char} & 1 byte  8 bits. \\ 
		 \texttt{char16\_t} & At least 16 bits \\ 
		 \texttt{char32\_t} & At least 32 bits. \\ 
		\hline
		
		 \texttt{signed char} 			& Min 8 bits. \\ 
		 \texttt{signed short int} 	& Min 16 bits. \\ 
		 \texttt{signed int} 			& Min 16 bits. \\ 
		 \texttt{signed long int} 		& Min 32 bits. \\
		 \texttt{signed long long int} & Min 64 bits. \\
		\hline
		
 		
		 \texttt{unsigned char} 			& Min 8 bits. \\ 
		 \texttt{" short int} 		& Min 16 bits. \\ 
		 \texttt{"int} 			& Min 16 bits. \\ 
		 \texttt{" long int} 		& Min 32 bits. \\
		 \texttt{" long long int} 	& Min 64 bits. \\
		\hline
	\end{tabular}
\end{center}
\subsection{Variablennamen}
Keine Leerzeichen, Satzzeichen oder \_ Symbole Keine Zahl oder am Anfang case sensitivity – Gross - Kleinschreibung beachten

\subsection{Einfache Variablen deklarieren}
\begin{lstlisting}
int a,a2;   int b (1);
int b = 10;  int b {1};
float c = a*b - 0.5;
\end{lstlisting}
\subsection{Casts}
Änderung einer Variable in einen anderen Type
\begin{lstlisting}	
double a = 1.5; int b;
b = int (a);
b = (int) a; // b=1
7/2 = 3 , 7/(double)2 = 7/2.0 = 3.5
double(7/2) = 3.0 , int(19/10.0) = 1
\end{lstlisting}


\subsection{Enum}
Enum ist ein Aufzählungstyp. Die Konstanten aus der Enum
kann man im Programm verwenden.
\begin{lstlisting}	
enum farbe {ROT, BLAU, GELB};
farbe f = ROT;
if(f != BLAU) { }; 
\end{lstlisting}

\subsection{Hexadezimaler Code \& Adressen}
0,1,...,9,A,B,C,D,E,F (hex) anstelle von 0,1,...,14,15,16 (dec)
Adressen werden hexadezimal angegeben. $a,a+1,a+2,a+3,...$

\begin{center}
	\begin{tabular}{ ll } 
		\hline
int,float(4byte) & double (8byte)\\
\hline
0x22ff70 & 0x22ff70\\
0x22ff74 & 0x22ff78\\
0x22ff78 & 0x22ff80\\
		\hline
	\end{tabular}
\end{center}

\subsection{Fliesskommazahlen}
\begin{center}
	\begin{tabular}{ ll } 
		float&1b$\rightarrow$sign, 8b$\rightarrow$exp, 23b$\rightarrow$mantisse\\
		&Wert = $(-1)^{\texttt{S}} \cdot 2^{(\texttt{E-127})} \cdot (1.\texttt{F})$\\
		&Bsp: $0.125 = 2^{3} \Rightarrow \texttt{S} \rightarrow 0, \texttt{E} \rightarrow 124, \texttt{F} \rightarrow 0$\\
		\hline
		 & 0 | 01111100 | 00000000...0 = 0.125\\
		 & 0 | 01111111 | 00000000...0 = 1\\
		 & 1 | 01111111 | 11000000...0 = -1.75\\
		 & 0 | 00000000 | 00000000...0 = 0\\
		 & 0 | 11111111 | 00000000...0 = +infty\\
		 & 0 | 00000000 | 00000000...0 = NaN\\
		\hline
		double& 1b$\rightarrow$sign, 11b$\rightarrow$exp, 52b$\rightarrow$mantisse\\
		& Wert = $(-1)^{\texttt{S}} \cdot 2^{(\texttt{E-1023})} \cdot (1.\texttt{F})$\\
	\end{tabular}
\end{center}















