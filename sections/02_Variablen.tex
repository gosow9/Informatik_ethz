\section{Variablen}

\begin{center}
	\begin{tabular}{ |l|l|l|l| } 
		\hline
		\textbf{Group} & \textbf{Type names} & \textbf{Notes on size / precision} \\
		\hline
		\multirow{3}{4em}{Character types} 
		& \texttt{char} & One byte in size at least 8 bits. \\ 
		& \texttt{char16\_t} & At least 16 bits \\ 
		& \texttt{char32\_t} & At least 32 bits. \\ 
		& \texttt{wchar\_t} & Can represent the largest character set \\ 
		\hline
		
		\multirow{3}{4em}{Integer types (signed)} 
		& \texttt{char} & One byte in size at least 8 bits. \\ 
		& \texttt{char16\_t} & At least 16 bits \\ 
		& \texttt{char32\_t} & At least 32 bits. \\ 
		\hline
	\end{tabular}
\end{center}
Mögliche Initialsisations von vaiablen
\begin{lstlisting}	
int x;
int x = 1;
int x (1);
int x {1};
\end{lstlisting}