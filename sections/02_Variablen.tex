\section{Variablen}

\begin{center}
	\begin{tabular}{ |l|l|l|l| } 
		\hline
		\textbf{Group} & \textbf{Type names} & \textbf{Notes on size / precision} \\
		\hline
		\multirow{3}{4em}{Character types} 
		& \texttt{char} & One byte  8 bits. \\ 
		& \texttt{char16\_t} & At least 16 bits \\ 
		& \texttt{char32\_t} & At least 32 bits. \\ 
		& \texttt{wchar\_t} & Largest character set \\ 
		\hline
		
		\multirow{3}{4em}{Integer types (signed)} 
		& \texttt{signed char} 			& Min 8 bits. \\ 
		& \texttt{signed short int} 	& Min 16 bits. \\ 
		& \texttt{signed int} 			& Min 16 bits. \\ 
		& \texttt{signed long int} 		& Min 32 bits. \\
		& \texttt{signed long long int} & Min 64 bits. \\
		\hline
		
 		\multirow{3}{4em}{Integer types (unsigned)} 
		& \texttt{unsigned char} 			& Min 8 bits. \\ 
		& \texttt{" short int} 		& Min 16 bits. \\ 
		& \texttt{"int} 			& Min 16 bits. \\ 
		& \texttt{" long int} 		& Min 32 bits. \\
		& \texttt{" long long int} 	& Min 64 bits. \\
		\hline
	\end{tabular}
\end{center}

Mögliche Initialsisations von vaiablen
\begin{lstlisting}	
int x;
int x = 1;
int x (1);
int x {1};
\end{lstlisting}