\section{Ein- und Ausgabe}
	Um die \lc{C++} Ein- und Ausgaben nutzen zu k�nnen, muss man die Bibliothek iostream einbinden. Das geschieht mit:
	\lstinputlisting{code/iostream.cpp}
	Damit die Ein- und Ausgabebefehle auch wirklich genutzt werden k�nnen, m�ssen sie mithilfe von
	\lstinputlisting{code/using_namespace_std.cpp}
	noch bekanntgegeben werden.\newline	
	\begin{minipage}[t]{11 cm}		
		\subsection{Streamkonzept}
			\begin{compactitem}
				\item Ein \lc{Stream} repr�sentiert einen sequentiellen Datenstrom.
				\item \lc{C++} stellt 4 Standardstreams zur Verf�gung:
					\begin{compactitem}
						\item \lc{cin}: Standard-Eingabesteam
						\item \lc{cout}: Standard-Ausgabestream
						\item \lc{cerr}: Standard-Fehlerausgabestream
						\item \lc{clog}: mit \lc{cerr} gekoppelt
					\end{compactitem}
				\item Alle diese Streams k�nnen auch mit einer Datei verbunden werden.	
				\item Alle Schl�sselw�rter m�ssen immer ganz links auf der Zeile stehen!\newline
			\end{compactitem} 
	\end{minipage}
	\hspace*{0.5cm}	
	\begin{minipage}[t]{7 cm}
		\subsection{Eingabebeispiele}
			\lstinputlisting{code/eingabe_3.cpp}	
	\end{minipage}
	\hspace{0.5cm}
	\subsection{Ausgabebeispiele}
		\lstinputlisting{code/ausgabe_3.cpp}
		\lstinputlisting{code/ausgabe_4.cpp}
		
