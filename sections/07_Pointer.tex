\section{Pointer und Referenzen} 
Bei Variablenübergabe (call by value) werden Kopien übergeben, welche nicht verändert werden können.\\
Bei Referenzübergabe (call by reference) kann die Subroutine die Werte bleibend verändern. \\
\textbf{Objekte einer Klasse und Strukturvariablen sollen immer by reference übergeben werden!} \\
\subsection{call by reference}
\vspace{-13pt}
\begin{multicols}{2}
	statisch:
	\begin{lstlisting}
		void swap(int& a, int& b){
			int tmp = a;
			a = b;
			b = tmp;
		}
		int main(){
			int x = 4;
			int y = 3;
			swap(x, y);// OK!
			return 0;
		}	
	\end{lstlisting}
	dynamisch:
	\begin{lstlisting}
		void swap(int* a, int* b){
			int tmp = *a;
			*a = *b;
			*b = tmp;
		}
		int main(){
			int x = 4;
			int y = 3;
			swap(&x, &y);// OK!
			return 0;
		}
	\end{lstlisting}
\end{multicols}
\subsection{call by value}
	\begin{lstlisting}
		void swap(int a, int b){
			int tmp = a;
			a = b;
			b = tmp;
		}
		int main(){
			int x = 4;
			int y = 3;
			swap(x, y); // keine Auswirkung
			return 0;
		}	
	\end{lstlisting}
\subsection{return by reference}
\begin{lstlisting}
	int& inc(int& i){
		return ++i;
	}	
\end{lstlisting}
Der Funktionsaufruf ist nun selbst ein L-Wert, was nun Ausdrücke wie \texttt{inc(inc(x))} oder \texttt{++inc(x)} erlaubt. \textbf{Achtung} Gültigkeitsbereiche: Return by reference auf lokale Variable ist undefined behavior.
\subsection{Pointer}
Ein Ausdruck vom Typ \texttt{T*} heisst Zeiger (Auf T). Der Wert eines Zeigers ist die Adresse, auf die er zeigt. Man kommt auf die Adresse eines Objektes entweder direkt beim Erzeugen mittels \texttt{new} oder mit dem Adress-Operator \texttt{\&}.Um undefiniertes Verhalten bei der Initialisierung zu vermeinden, erstellt man einen Null-Zeiger.
\begin{lstlisting}
	int* r; //undefined behaviour
	int* q=nullptr; //(zeigt explizit ins nichts)
	
	int i = 5;
	int* p = &i; //Zeiger auf i
	std::cout<<*p; // Output: 5
\end{lstlisting}
\includegraphics[width=0.24 \textwidth]{sections/pointer}
Um nun der Wert bei der Adresse des Pointers zu erhalten braucht man den Deferenz-Operator \texttt{*}.
\subsubsection{Pointerarithmetik}
Mit Pointer-




